\section{Introduction}
In part, implications in formal concept analysis presents a method of determining correspondence between sets of attributes. A formal context describes a Galois connection between sets of attributes and sets of objects - as such, correspondence between attributes equally describes the correspondence between objects which have said attributes. That is, if $\texttt{wings} \rightarrow \texttt{flies}$ is an accepted implication then all those objects which have \texttt{wings} also \texttt{fly}.

It is immediately apparent that this kind of reasoning has limitations: \textit{an ostritch has wings but does not fly}. Either the implication must be rejected, or amended to account for the existence of ostritches $\texttt{wings}$ and $\neg \texttt{ostritch} \rightarrow \texttt{flies}$ - but what then about penguins? It is not practical to encode every exception to some conditional statement in the statement itself. Equally, discarding the rule entirely is an unappealing prospect since the connection between having wings and flying is of obvious utility.

The rest of this paper is structured as follows: \autoref{section—fca} introduces the necessary definitions and concepts from FCA, \autoref{section—nmr} then provides an account of preferential and rational consequence relations, as well as a brief exposition of entailment - note \autoref{section—nmr} uses the language of propositional logic when explaining these concepts, while the remainder of the paper uses the attribute logic of FCA. \autoref{section—RC_fca} details our approach to implement non-monotonic entailment to FCA.